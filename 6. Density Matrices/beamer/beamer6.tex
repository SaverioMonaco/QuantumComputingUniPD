\documentclass[pt12]{beamer}
%\documentclass[pt12,externalviewer]{beamer}

\usepackage{amssymb}
\usepackage{minted}
%\usepackage{rotating}
%\usepackage{amsmath}
%\usepackage{tikz}
%\usepackage{beamergraphics}
\usepackage[latin1]{inputenc}
\usepackage[T1]{fontenc}
\usepackage{multicol}
\usepackage{svg}
%\usepackage[absolute,overlay]{textpos}
%\usepackage{pdfcolparallel}

%\usepackage{tcolorbox}
%\tcbuselibrary{fitting}

\definecolor{PDred}{HTML}{9A001B}

\mode<presentation>
{
  \usetheme{Warsaw} 
%  \usetheme{Madrid}
%  \usetheme{Montpellier}
%  \usetheme{Marburg} 
  \usecolortheme[named=PDred]{structure}
  \setbeamercolor{alerted text}{fg=PDred}
  \setbeamercovered{transparent}
  \setbeamertemplate{section in toc}[ball unnumbered]
}

%\setbeamertemplate{footline}{\hfill\insertframenumber/\inserttotalframenumber} 

\expandafter\def\expandafter\insertshorttitle\expandafter{%
  \insertshorttitle\hfill%
  \insertframenumber\,/\,\inserttotalframenumber}

\newcommand{\backupbegin}{
   \newcounter{framenumberappendix}
   \setcounter{framenumberappendix}{\value{framenumber}}
}
\newcommand{\backupend}{
   \addtocounter{framenumberappendix}{-\value{framenumber}}
   \addtocounter{framenumber}{\value{framenumberappendix}} 
}

\newcommand{\refer}[1]{%
   \begin{flushright}
      {\alert{\tiny #1}}
   \end{flushright}}
  
\newcommand{\lrefer}[1]{%
   \begin{flushleft}
      {\alert{\tiny #1}}
   \end{flushleft}}
  
\newcommand{\param}[1]{%
   \begin{flushright}
      {\small #1}
   \end{flushright}
   \vspace{-1.5\baselineskip}
}

\newcommand{\sech}{\mathop{\rm sech}\nolimits}
\newcommand{\sgn}{\mathop{\rm sgn}\nolimits}
\newcommand{\etal}{{\em et al.}}


\title[Density Matrices \& QuBits]{Density Matrices \& QuBits}

\author[Saverio Monaco]{\large{Saverio Monaco}\bigskip\\ Quantum Information and Computing}
\date{\today}
\institute[DFA.UniCT]{
\begin{minipage}[c]{1.3truecm}
\includegraphics[width=\textwidth]{../../teximgs/PODlogo}
\end{minipage}
\begin{minipage}[c]{4.7truecm}
\begin{flushleft}
\begin{sl}
Dipartimento di Fisica e Astronomia\\ 
``Galileo Galilei''
\end{sl}
\end{flushleft}
\end{minipage}
\begin{minipage}[c]{3.2truecm}
\includegraphics[width=\textwidth]{../../teximgs/800anni_logo}
\end{minipage}}


\begin{document}

\begin{frame}[plain]
\titlepage
\end{frame}

\newcommand{\ket}[1]{\left|#1\right>}
\newcommand{\bra}[1]{\left<#1\right|}

\begin{frame}[label=Theory]
\frametitle{Theory}
\framesubtitle{Wavefunction}
\tableofcontents[pausesections]
\textbf{Goal of the exercise:} numerical handling of pure N-body wavefunctions.\bigskip\\
Assume N particles each described by a d-dimensional Hilbert space $\left(\varphi_i\in\mathcal{H}^d\right)$.\medskip\\
There are two possible systems:
\begin{itemize}
	\item (general) \textbf{Unseparable case:} 
	$$\Psi = \sum_{a_1,a_2,...,a_N}C_{a_1,a_2,...,a_N}\ket{a_1}\otimes\ket{a_2}\otimes...\otimes\ket{a_N}\qquad dim\equiv d^N\hspace*{6cm}\text{}$$
	\item \textbf{Separable case:} (no interactions between particles)
	$$\Psi = \bigotimes_{i=1}^N\varphi_i\qquad dim\equiv d\times N\hspace*{6cm}\text{}$$
\end{itemize}
\end{frame}

\begin{frame}[label=Theory]
	\frametitle{Theory}
	\framesubtitle{Density matrix}
	\tableofcontents[pausesections]
	It is possible to construct an operator $\hat{\rho}$ out of the total wavefunction $\Psi$ described before:
	$$\rho = \ket{\Psi}\bra{\Psi} = \sum_{a_1,..,a_N}\sum_{a_1^\prime,..,a_N^\prime}\,C_{a_1,..,a_N}C_{a_1^\prime,..,a_N^\prime}\ket{a_1,..,a_N}\bra{a_1^\prime,..,a_N^\prime}$$
	\textbf{Properties:}
	\begin{itemize}
		\item hermitian $\leftrightarrow\,\,\, \rho = \rho^\dagger$
		\item $Tr(\rho) = 1$
		\item Eigenvalues of $\rho$ are: $\{1,0,...,0\}$
    \end{itemize}
    Other properties:
    \begin{itemize}
    	\item $\rho^2 = \rho$
    	\item $<\rho> \,= Tr(\rho^2) \leq 1$
    \end{itemize}
\end{frame}

\begin{frame}[fragile,label=Code development]
	\frametitle{Code development}
	\framesubtitle{Derived type and related functions}
	\tableofcontents[pausesections]
	\textbf{Custom derived type:}\vspace{-1cm}\\
	\begin{figure}[!h]\begin{minted}[fontsize=\footnotesize, frame=lines, framesep=4mm, gobble=2]{Fortran}
		type qsystem
		  integer :: N ! Number of systems
		  integer :: d ! number of states
		  logical :: separability ! Wether the whole system is separable
		  double complex, dimension(:), allocatable   :: PSI ! Total WF
		  double complex, dimension(:,:), allocatable :: rho ! Density matrix
		end type qsystem
	\end{minted}
    \end{figure}
	\textbf{Related functions:}\vspace{-1cm}\\
	\begin{figure}[!h]\begin{minted}[fontsize=\footnotesize, frame=lines, framesep=4mm, gobble=3]{Fortran}
			function densmat_pure_init(N, d, SEP, DEBUG) result(system)
			subroutine densmat_genstates(system)
			subroutine densmat_readcoeffs(system)
			function densmat_computerho1(system,d) result(rho1)
		\end{minted}
	\end{figure}
	
\end{frame}

\begin{frame}[fragile,label=Code development]
	\frametitle{Code development}
	\framesubtitle{Indices}
	\tableofcontents[pausesections]
	A explicit way to visualize the total wavefunction (non-separable case):
	
	$$\begin{align}
	\Psi = &a_0\ket{0,0,...,0} + a_1 \ket{0,0,...,1} + ... \\
	     &... + a_{d-1}\ket{0,0,...,d-1} + a_d \ket{0,0,...,1,0}+ ...
	\end{align}$$

	Getting the index of the array from the eigenvectors combination and viceversa is just a \textbf{change of base}:
	
	\begin{multicols}{2}
		\begin{figure}[!h]\begin{minted}[fontsize=\tiny, framesep=4mm, gobble=3]{Fortran}
			function basechange_to(b_to, number, N) 
			                       result(number_b_to)
			                       
			integer :: b_to, number, N, ii
			integer, dimension(N) :: number_b_to
			
			number_b_to = 0*number_b_to
			do ii = 1, N, 1
			  number_b_to(N - ii + 1) = modulo(number, b_to)
			  number = number/b_to
			end do
			end function	
			\end{minted}
		\end{figure}
		\columnbreak
		\begin{figure}[!h]\begin{minted}[fontsize=\tiny, framesep=4mm, gobble=3]{Fortran}
			function basechange_from(b_from, number_from, N) 
			                         result(number_b10)
			                         
			integer :: b_from, number_b10, N, ii
			integer, dimension(N) :: number_from
			
			do ii = 1, N, 1
			  number_b10 = number_b10 + 
			               number_from(N - ii + 1)*
			               b_from**(ii - 1)
			end do
			end function
			\end{minted}
		\end{figure}
	\end{multicols}
\end{frame}

\begin{frame}[label=Results]
	\frametitle{Results}
	\tableofcontents[pausesections]
	\begin{itemize}
		\item \textbf{Correctness}: Trace and eigenvalues of $\rho$, (and it's partial trace as well) were computed to check the correctness. The results given match the theory up to some numerical error.
		\item \textbf{Stability}: The program may give and error when trying to allocate too much memory. The number of elements of $\rho$ is $d^{2N}$ and each element occupies 16 bytes (\texttt{double complex})
		\begin{figure}[h]
		\centering
		\hspace{-1cm}\includesvg[width=.6\textwidth]{../imgs/maxalloc.svg}
		\end{figure}
	\end{itemize}
	\tiny{Data was generated using a Ubuntu machine with 8GB of RAM}
\end{frame}

\include{sslide}

\end{document}

